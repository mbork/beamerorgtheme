%% beamercolorthemeorg.sty
%%
%% Copyright 2007 by Till Tantau and 2013 by Marcin Borkowski
%%
%% This beamer theme is based on Till Tantau's default Beamer theme,
%% with modifications by Marcin Borkowski.
%
% This work may be distributed and/or modified under the
% conditions of the LaTeX Project Public License, version 1.3c,
% found in the file lppl.txt.
%
% This work has the LPPL maintenance status `maintained'.
% 
% The Current Maintainer of this work is Marcin Borkowski <mbork@mbork.pl>.
%
% This work consists of the files:
%   beamercolorthemeorg.sty
%   beamerfontthemeorg.sty
%   beamerinnerthemeorg.sty
%   beamerouterthemeorg.sty
%   beamerthemeorg.sty
%   beamerthemeorg-demo.tex

\documentclass{beamer}

\usetheme{org}
\setbeamertemplate{itemize subitem}[plus]

\begin{document}

\author{Marcin Borkowski}
\title{An example presentation}
\subtitle{using the \texttt{beamer} package and the \orgbf{org} theme}
\date{\today}
\institute{\texttt{http://mbork.pl}}

\begin{frame}
  \maketitle
\end{frame}

\begin{frame}
  \frametitle{Contents}
  \tableofcontents
\end{frame}

\section{First section}

\subsection{Subsection}
\begin{frame}
  An example frame.
\end{frame}

\begin{frame}
  \frametitle{A frame with a definition and an example}
  \begin{definition}
    Very \alert{cool} definition.
  \end{definition}
  \begin{example}
    And a~fine example.
  \end{example}
  \begin{alertblock}{Look here!}
    And this is an \orgit{alert block}.
  \end{alertblock}
\end{frame}

\begin{frame}[label=enumerations]
  \frametitle{Enumerations}
  \begin{itemize}[<+->]
  \item qwerty
  \item asdf
    \begin{itemize}[<+->]
    \item as
    \item df
    \end{itemize}
  \item zxcvbnm
    \begin{itemize}[<+->]
    \item 1234
      \begin{itemize}[<+->]
      \item 567
      \end{itemize}
    \end{itemize}
  \end{itemize}
\end{frame}

\subsection{Second subsection}
\begin{frame}
  \frametitle{A frame with a theorem}
  \begin{theorem}
    A cool theorem.
    \begin{equation*}
      a^2+b^2=c^2\cdot\int_0^\infty f(x)\,dx
    \end{equation*}
  \end{theorem}  
  \pause
  \begin{proof}
    With a cool proof.
  \end{proof}
\end{frame}

\end{document}
